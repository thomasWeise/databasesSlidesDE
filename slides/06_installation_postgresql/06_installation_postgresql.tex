\pdfminorversion=7%
\documentclass[aspectratio=169,mathserif,notheorems]{beamer}%
%
\xdef\bookbaseDir{../../bookbase}%
\xdef\sharedDir{../../shared}%
\RequirePackage{\bookbaseDir/styles/slides}%
\RequirePackage{\sharedDir/styles/styles}%
\toggleToGerman%
%
\subtitle{6.~Installation von PostgreSQL}%
%
\begin{document}%
%
\startPresentation%
%
\section{Einleitung}%
%
\begin{frame}%
\frametitle{Einleitung}%
\begin{itemize}%
\item In unserem Kurs werden wir das Open Source DBMS \postgresql\cite{TA2024DDAMWPAM,FP2023LP,OH2017PUAR,B2024PELUYDW} verwenden, um mit Datenbanken zu experimentieren.%
\item<2-> Sie sollten es auf Ihren persönlichen Computer installieren.%
\end{itemize}%
\locateGraphic{}{width=0.225\paperwidth}{../05_software_und_literatur/graphics/postgresqlLogo}{0.7}{0.58}%
\end{frame}%
%
\begin{frame}
\frametitle{PostgreSQL}%
\begin{itemize}%
\item \postgresql\ basiert auf der Klient-Server-Architektur.%
\item<2-> Das DBMS ist als Server-Programm implementiert.%
\item<3-> Es managed die Datenbanken, speichert die Daten, und stellt sie auch zur Verfügung.%
\item<4-> Andere Programme können sich~(als Klienten) mit dem Server verbinden, um auf die Datenbanken zuzugreifen.%
\item<5-> \postgresql\ stellt auch bereits ein Klient-Programm zur Verfügung, nämlich \psql.%
\item<6-> Mit \sql\ können  menschliche Benutzer mit dem DBMS-Server über das Terminal kommunizieren.%
\item<7-> Wir installieren sowohl den Server als auch den Klienten.%
\item<8-> Mehr dazu können Sie auf der \postgresql\ Downloadseite \url{https://www.postgresql.org/download} finden.%
\item<9-> Hier bieten wir Schritt-für-Schritt Anleitungen für \ubuntu\ \linux\ und \microsoftWindows\ an.%
\end{itemize}%
\end{frame}%
%
\section{Ubuntu Linux}%
%
\begin{frame}[t]%
\frametitle{Installation von PostgreSQL unter Ubuntu Linux}%
\begin{itemize}%
%
\only<-1>{\item Wir installieren \postgresql\ mit Hilfe des \bashil{apt-get install} Kommandos in einem Terminal das mit \ubuntuTerminal\ geöffnet wird.}%
%
\only<2>{\item Dieser Befehl benötigt das Super-User-Password. %
Wir schreiben es und drücken~\keys{\enter}.}%
%
\only<3>{\item Wir werden gefragt, ob wie die benötigten Pakete wirklich installieren wollen.}%
%
\only<4>{\item Wir bejahen das mit~\keys{y+\enter}.}%
%
\only<5>{\item Die Installation läuft ab und endet erfolgreich.}%
%
\only<6>{\item Wir wollen nun den Status der frischen \postgresql\ Installation prüfen. %
Das können wir durch das Kommando \bashil{systemctl status postgresql} tun, welches wir mit~\keys{\enter} ausführen.}%
%
\only<7>{\item Die Ausgabe zeigt uns, dass der \postgresql-Service nun läuft. %
Er wird immer starten, wenn wir unser System booten.}%
%
\only<8>{\item Wir drücken~\keys{q+\enter} um das Kommande zu beenden.}%
%
\only<9>{\item \psql\ ist das Klientenprogramm, mit dem wir uns auf den \postgresql-Server verbinden können. %
Es wurde ebenfalls installiert. %
Wir können seine Version mit dem Kommando \bashil{psql --version} herausfinden, welches wir mit~\keys{\enter} ausführen.}%
%
\only<10>{\item Beim Erstellen dieser Slides war Version~16.6 aktuell.}%
%
\only<11>{\item Um ein Passwort für den \postgresql-Benutzer \textil{postgresql} zu setzen, müssen wir uns in \psql\ mit \bashil{sudo}-Privilegien einwählen, allerdings unter dem neu erstellten Systemnutzer \textil{postgres}. %
Wir tuen dies durch \bashil{sudo -u postgresql psql} und drücken~\keys{\enter}.}%
%
\only<12>{\item Wir müssen das Passwort gegebenenfalls zweimal eintippen.}%
%
\only<13>{\item \psql\ verbindet sich nun zu unserem lokalen \postgresql-server.}%
%
\only<14>{\item Wir können nun mit dem DBMS über SQL kommunizieren. %
Wir nutzen den Befehl \sqlil{ALTER USER postgres PASSWORD 'XXX';}, wobei wir \textil{XXX} natürlich mit einem sicheren Password ersetzen~(im Screenshot covered).}%
%
\only<15>{\item Wir drücken~\keys{\enter} und das System bestätigt die Änderung, in dem es das Kommando \sqlil{ALTER ROLE} nochmal ausgibt. %
Von nun an hat das Hauptbenutzerkonto des Servers ein sicheres Passwort.}%
%
\only<16>{\item Wir verlassen \psql, in dem wir \keys{\textbackslash+q+\enter} schreiben.}%
%
\only<17->{\item Wir sind fertig.}%
%
\end{itemize}%
%
\locateGraphic{1}{width=0.64\paperwidth}{graphics/ubuntu/installingPostgresUbuntu01aptGet}{0.18}{0.33}%
\locateGraphic{2}{width=0.64\paperwidth}{graphics/ubuntu/installingPostgresUbuntu02pass}{0.18}{0.33}%
\locateGraphic{3}{width=0.64\paperwidth}{graphics/ubuntu/installingPostgresUbuntu03yn}{0.18}{0.33}%
\locateGraphic{4}{width=0.64\paperwidth}{graphics/ubuntu/installingPostgresUbuntu04yny}{0.18}{0.33}%
\locateGraphic{5}{width=0.64\paperwidth}{graphics/ubuntu/installingPostgresUbuntu05install}{0.18}{0.33}%
\locateGraphic{6}{width=0.64\paperwidth}{graphics/ubuntu/installingPostgresUbuntu06systemctlCheckStatus}{0.18}{0.33}%
\locateGraphic{7}{width=0.64\paperwidth}{graphics/ubuntu/installingPostgresUbuntu07systemctlCheckStatusRes}{0.18}{0.33}%
\locateGraphic{8}{width=0.64\paperwidth}{graphics/ubuntu/installingPostgresUbuntu08systemctlCheckStatusQ}{0.18}{0.33}%
\locateGraphic{9}{width=0.64\paperwidth}{graphics/ubuntu/installingPostgresUbuntu09psqlVersionA}{0.18}{0.33}%
\locateGraphic{10}{width=0.64\paperwidth}{graphics/ubuntu/installingPostgresUbuntu10psqlVersionB}{0.18}{0.33}%
\locateGraphic{11}{width=0.64\paperwidth}{graphics/ubuntu/installingPostgresUbuntu11sudoPsql}{0.18}{0.33}%
\locateGraphic{12}{width=0.64\paperwidth}{graphics/ubuntu/installingPostgresUbuntu12sudoPsqlPw}{0.18}{0.33}%
\locateGraphic{13}{width=0.64\paperwidth}{graphics/ubuntu/installingPostgresUbuntu13PsqlPrompt}{0.18}{0.33}%
\locateGraphic{14}{width=0.64\paperwidth}{graphics/ubuntu/installingPostgresUbuntu14PsqlAlterUserPassword}{0.18}{0.33}%
\locateGraphic{15}{width=0.64\paperwidth}{graphics/ubuntu/installingPostgresUbuntu15PsqlAlterUserPasswordDone}{0.18}{0.33}%
\locateGraphic{16}{width=0.64\paperwidth}{graphics/ubuntu/installingPostgresUbuntu16PsqlQuit}{0.18}{0.33}%
\locateGraphic{17}{width=0.64\paperwidth}{graphics/ubuntu/installingPostgresUbuntu17Done}{0.18}{0.33}%
%
\end{frame}%
%
\section{Microsoft Windows}%
%
\begin{frame}[t]%
\frametitle{Installation von PostgreSQL unter Microsoft Windows}%
\begin{itemize}%
\only<-1>{\item Wir besuchen die Webseite \url{https://www.postgresql.org/download} und clicken auf~\menu{Windows}.}%
%
\only<2>{\item Wir klicken auf den Link \emph{download the installer}, der uns zu \url{https://www.enterprisedb.com/downloads/postgres-postgresql-downloads} bringt.}%
%
\only<3>{\item Hier gibt es eine große Auswahl verschiedener Betriebssysteme. %
Wir suchen die neueste Version~(die oberste) für \microsoftWindows. %
Als ich diese Slides gemacht habe, war das Version~17.2. %
Wir klicken auf das download-Icon.}%
%
\only<4>{\item Der Download beginnt.}%
%
\only<5>{\item Nach dem der Download fertig ist, müssen wir die Datei finden\dots}%
%
\only<6>{\item {\dots}und dann ausführen, durch Klick auf~\menu{Open file}.}%
%
\only<7>{\item Wenn wir gefragt werden, ob wir es dem Download erlauben wollen, unsere Maschne zu verändern, dann klicken wir auf~\menu{Yes}.}%
%
\only<8>{\item Der Installer beginnt seine Arbeit.}%
%
\only<9>{\item Im Welcome-Bildschirm klicken wir auf~\menu{Next}.}%
%
\only<10>{\item Wir können nun den Ordner auswählen, in den \postgresql\ installiert werden soll. %
Wir lassen die Standardeinstellungen unverändert und klicken auf~\menu{Next}.}%
%
\only<11>{\item Wir können auswählen, was installiert werden soll. %
Wir lassen die Standardeinstellungen unverändert und klicken auf~\menu{Next}.}%
%
\only<12>{\item Wir können auswählen, wo die Datenbanken gespeichert werden sollen. %
Wir lassen die Standardeinstellungen unverändert und klicken auf~\menu{Next}.}%
%
\only<13>{\item Nun müssen wir ein sicheres Passwort für \postgresql\ angeben. %
Bitte wählen Sie bedachtsam ein Passwrt, merken Sie es sich gut, und geben es in die beiden Formularfelder ein. %
Dann klicken wir~\menu{Next}.}%
%
\only<14>{\item Wir können einen Port auswählen, an dem der \postgresql-Server auf Verbindungen warten soll. %
Wir lassen die Standardeinstellung~5432 unverändert und klicken auf~\menu{Next}.}%
%
\only<15>{\item Mit der \pgls{locale} können wir länder- und kulturspezifische Einstellungen laden, z.B. Nummern- und Währungsformate. %
Wir lassen die Standardeinstellungen unverändert und klicken auf~\menu{Next}.}%
%
\only<16>{\item Wir werden über die Komponenten informiert, die nun installiert werden. %
Wir klicken~\menu{Next}.}%
%
\only<17>{\item Wir werden gefragt, ob wir bereit für die Installation sind. %
Wir klicken~\menu{Next}.}%
%
\only<18>{\item Die Installation beginnt.}%
%
\only<19>{\item Die Installation schreitet voran.}%
%
\only<20>{\item Nachdem die Installation fertig ist, werden wir gefragt, ob wir die \emph{Stack Builder}-Software benutzen wollen, um weitere Komponenten zu installieren. %
Wir wollen nicht und un-markieren die entsprechende Box. %
Wir klicken auf~\menu{Finish}. %
\postgresql\ ist nun installiert.}%
%
\only<21>{\item Um die Installation zu testen, öffnen wir ein Terminal. %
Dafür drücken wir die Tasten~\keys{\OSwin + R}.}%
%
\only<22>{\item Wir schreiben \textil{cmd} in die Eingabebox und drücken~\keys{\enter}.}%
%
\only<23>{\item Ein neues Terminalfenster öffnet sich.}%
%
\only<24>{\item Wir geben den Pfad zum \batil{bin}-Ordner in dem Ordner ein, in dem  wir \postgresql\ installiert haben. %
Unter den Standardeinstellungen wäre das \batil{cd "C:\\Program Files\\PostgreSQL\\bin"}. %
Wir drücken~\keys{\enter}.}%
%
\only<25>{\item Wir sind nun in diesem Ordner.}%
%
\only<26>{\item Nun wollen wir die Version des \psql-Klienten herausfinden. %
Wir tuen dies in dem wir \batil{psql -V} schreiben und dann \keys{\enter}~drücken.}%
%
\only<27>{\item In meinem Fall zeigt die Ausgabe, dass Version~17.2 installiert wurde.}%
%
\only<28>{\item Wir wollen nun sehen welche Version des \postgresql-Servers installiert wurde. %
Damit testen wir gleichzeitig, ob die Installation geklappt hat. %
Darum schreiben wir nun \batil{psql -U postgres}, starten also \psql\ als Benutzer \textil{postgres}.}%
%
\only<29>{\item Bei Programmstart müssen wir nun das Passwort für den Benutzer \textil{postgres} eingeben. %
Das ist das Passwort, das wir bei der Installation angegeben hatten. %
Wir geben es ein und drücken~\keys{\enter}. %
Wir sind nun in der \psql-Konsole und sehen den \textil{postgres=\#}-Prompt.}%
%
\only<30>{\item Wir geben das SQL-Kommando \sqlil{SELECT * FROM VERSION();} ein und drücken~\keys{\enter}.}%
%
\only<31>{\item Das Ergebnis zeigt in meinem Fall, das der \postgresql-Server auch Version~17.2 hat.%
Nun schreiben Sie \keys{\textbackslash+q+\enter}, wodurch \psql\ verlassen wird.}%
%
\only<32>{\item Wir erforschen nun, wie der \postgresql-Server auf unserer \microsoftWindows\ läuft: %
Er wird als \emph{Service}, i.e., als \emph{Dienst} gestartet. %
Wir drücken~\keys{\OSwin}, schreiben \textil{services}, und klicken auf das \inQuotes{Zahnrad-Symbol} mit Namen \emph{Services} das erscheint.}%
%
\only<33>{\item Das \emph{Services} Systemfenster öffnet sich. %
Wir suchen nach einem Service, dessen Name nach \postgresql\ klingt. %
In meinem Fall ist das \textil{postgresql-x64-17}. %
Wir rechsts-klicken auf ihn.}%
%
\only<34>{\item Im erscheinenden Pop-up-Menü klicken wir auf~\menu{Properties}.}%
%
\only<35>{\item Der Diensteigenschaften-Dialog erscheint.}%
%
\only<36>{\item Wir klicken auf die Drop-Down-Box~\menu{Startup type:}. %
Sie steht auf~\menu{Automatic}, was bedeutet, das \postgresql\ immer gestartet wird, wenn Ihr System started.}%
%
\only<37>{\item \alert{Die folgenden Schritte sind optional}.%
Wenn Sie nicht wollen, das \postgresql\ bei jedem Systemstart startet, dann können Sie das ausstellen. %
Sie müssen dafür \menu{Manual} als \menu{Startup type:} auswählen und auf \menu{Apply} klicken.}%
%
\only<38>{\item Der \postgresql-Service läuft dann zwar aktuell noch, aber wird nicht bei Systemstart mit gestartet. %
Sie können das wieder rückgängig machen, in dem Sie wieder \menu{Automatic} als \menu{Startup type:} auswählen. %
Sie können den aktuell laufenden Service auch stoppen, in dem Sie auf \menu{Stop} klicken.}%
%
\only<39>{\item Dann wird der \postgresql\ service angehalten und der Server läuft nicht mehr.}%
%
\only<40>{\item Jetzt läuft der Service nicht mehr. %
Wir starten ihn wieder, in dem wir auf den \menu{Start}-Button drücken.}%
%
\only<41>{\item Nun läuft der Service wieder.}%
%
\only<42>{\item Wir klicken auf \menu{OK} und verlassen den Dialog.}%
%
\only<43->{\item Wir sehen, dass der Service läuft (ist~\emph{Running}) und im Modus~\emph{Manual}~(wenn wir diesen Modus ausgewählt hatten). %
Wenn wir herunterfahren, wird das DBMS angehalten. %
Bei einem Neustart startet es nicht automatisch~(es sei denn, Sie haben \emph{Automatic} als \emph{Startup type} eingestellt). %
Wenn Sie damit arbeiten wollen, dann müssten Sie wieder den \emph{Services} betreten und den Service manuell starten.}%
\end{itemize}%
%
\locateGraphicTB{1}{width=0.64\paperwidth}{graphics/windows/installingPostgresWindows01website}{0.18}{0.33}%
\locateGraphicTB{2}{width=0.64\paperwidth}{graphics/windows/installingPostgresWindows02downloadWindows}{0.18}{0.33}%
\locateGraphicTB{3}{width=0.64\paperwidth}{graphics/windows/installingPostgresWindows03ebdWebsite}{0.18}{0.33}%
\locateGraphicTB{4}{width=0.64\paperwidth}{graphics/windows/installingPostgresWindows04download}{0.18}{0.33}%
\locateGraphicTB{5}{width=0.64\paperwidth}{graphics/windows/installingPostgresWindows05downloadsA}{0.18}{0.33}%
\locateGraphicTB{6}{width=0.64\paperwidth}{graphics/windows/installingPostgresWindows06downloadsB}{0.18}{0.33}%
\locateGraphicTB{7}{width=0.5\paperwidth}{graphics/windows/installingPostgresWindows07install}{0.25}{0.29}%
\locateGraphicTB{8}{width=0.5\paperwidth}{graphics/windows/installingPostgresWindows08installerStarts}{0.25}{0.33}%
\locateGraphicTB{9}{width=0.5\paperwidth}{graphics/windows/installingPostgresWindows09wizardWelcome}{0.25}{0.29}%
\locateGraphicTB{10}{width=0.5\paperwidth}{graphics/windows/installingPostgresWindows10wizardDir}{0.25}{0.29}%
\locateGraphicTB{11}{width=0.5\paperwidth}{graphics/windows/installingPostgresWindows11wizardWhat}{0.25}{0.29}%
\locateGraphicTB{12}{width=0.5\paperwidth}{graphics/windows/installingPostgresWindows12wizardDataDir}{0.25}{0.29}%
\locateGraphicTB{13}{width=0.5\paperwidth}{graphics/windows/installingPostgresWindows13pw}{0.25}{0.29}%
\locateGraphicTB{14}{width=0.5\paperwidth}{graphics/windows/installingPostgresWindows14port}{0.25}{0.29}%
\locateGraphicTB{15}{width=0.5\paperwidth}{graphics/windows/installingPostgresWindows15locale}{0.25}{0.29}%
\locateGraphicTB{16}{width=0.5\paperwidth}{graphics/windows/installingPostgresWindows16preInstall}{0.25}{0.29}%
\locateGraphicTB{17}{width=0.5\paperwidth}{graphics/windows/installingPostgresWindows17ready}{0.25}{0.29}%
\locateGraphicTB{18}{width=0.5\paperwidth}{graphics/windows/installingPostgresWindows18installA}{0.25}{0.29}%
\locateGraphicTB{19}{width=0.5\paperwidth}{graphics/windows/installingPostgresWindows19installB}{0.25}{0.29}%
\locateGraphicTB{20}{width=0.5\paperwidth}{graphics/windows/installingPostgresWindows20finish}{0.25}{0.29}%
\locateGraphicTB{21}{width=0.5\paperwidth}{graphics/windows/installingPostgresWindows21windowsR}{0.25}{0.33}%
\locateGraphicTB{22}{width=0.5\paperwidth}{graphics/windows/installingPostgresWindows22cmd}{0.25}{0.33}%
\locateGraphicTB{23}{width=0.64\paperwidth}{graphics/windows/installingPostgresWindows23terminal}{0.18}{0.33}%
\locateGraphicTB{24}{width=0.64\paperwidth}{graphics/windows/installingPostgresWindows24cd}{0.18}{0.33}%
\locateGraphicTB{25}{width=0.64\paperwidth}{graphics/windows/installingPostgresWindows25inDir}{0.18}{0.33}%
\locateGraphicTB{26}{width=0.64\paperwidth}{graphics/windows/installingPostgresWindows26psqlV}{0.18}{0.33}%
\locateGraphicTB{27}{width=0.64\paperwidth}{graphics/windows/installingPostgresWindows27psqlVres}{0.18}{0.33}%
\locateGraphicTB{28}{width=0.64\paperwidth}{graphics/windows/installingPostgresWindows28psqlU}{0.18}{0.33}%
\locateGraphicTB{29}{width=0.64\paperwidth}{graphics/windows/installingPostgresWindows29pw}{0.18}{0.33}%
\locateGraphicTB{30}{width=0.64\paperwidth}{graphics/windows/installingPostgresWindows30selectVer}{0.18}{0.33}%
\locateGraphicTB{31}{width=0.64\paperwidth}{graphics/windows/installingPostgresWindows31verResQ}{0.18}{0.33}%
\locateGraphicTB{32}{width=0.5\paperwidth}{graphics/windows/installingPostgresWindows32services}{0.25}{0.33}%
\locateGraphicTB{33}{width=0.5\paperwidth}{graphics/windows/installingPostgresWindows33servicesPostgres}{0.25}{0.3}%
\locateGraphicTB{34}{width=0.5\paperwidth}{graphics/windows/installingPostgresWindows34servicesProperties}{0.25}{0.3}%
\locateGraphicTB{35}{width=0.35\paperwidth}{graphics/windows/installingPostgresWindows35servicesPropertiesPostgres}{0.325}{0.27}%
\locateGraphicTB{36}{width=0.35\paperwidth}{graphics/windows/installingPostgresWindows36servicesStartupType}{0.325}{0.27}%
\locateGraphicTB{37}{width=0.35\paperwidth}{graphics/windows/installingPostgresWindows37servicesManual}{0.55}{0.27}%
\locateGraphicTB{38}{width=0.32\paperwidth}{graphics/windows/installingPostgresWindows38servicesStop}{0.55}{0.31}%
\locateGraphicTB{39}{width=0.35\paperwidth}{graphics/windows/installingPostgresWindows39servicesStopping}{0.325}{0.27}%
\locateGraphicTB{40}{width=0.35\paperwidth}{graphics/windows/installingPostgresWindows40servicesStart}{0.325}{0.27}%
\locateGraphicTB{41}{width=0.35\paperwidth}{graphics/windows/installingPostgresWindows41servicesStarting}{0.325}{0.27}%
\locateGraphicTB{42}{width=0.35\paperwidth}{graphics/windows/installingPostgresWindows42servicesOK}{0.325}{0.27}%
\locateGraphicTB{43}{width=0.45\paperwidth}{graphics/windows/installingPostgresWindows43services}{0.275}{0.41}%
%
\end{frame}%
%
\section{Zusammenfassung}%
%
\begin{frame}%
\frametitle{Zusammenfassung}%
\begin{itemize}%
\item Nun haben Sie das \postgresql\ Datenbankmanagementsystem auf Ihrem Computer installiert.%
\item<2-> Damit können wir nun mit \inQuotes{echten} Datenbanken arbeiten.%
\item<3-> Cool.%
\end{itemize}%
\end{frame}%
%
\endPresentation%
\end{document}%%
\endinput%
%
