\pdfminorversion=7%
\documentclass[aspectratio=169,mathserif,notheorems]{beamer}%
%
\xdef\bookbaseDir{../../bookbase}%
\xdef\sharedDir{../../shared}%
\RequirePackage{\bookbaseDir/styles/slides}%
\RequirePackage{\sharedDir/styles/styles}%
\toggleToGerman%
%
\subtitle{5.~Software und Literatur}%
%
\begin{document}%
%
\startPresentation%
%
\section{Einleitung}%
%
\begin{frame}[t]%
\frametitle{Einleitung}%
\begin{itemize}\only<-3>{%
\item Heute gibt es eine große Auswahl an relationalen Datenbankmanagementsystemen.%
\item<2-> \citetitle{RS2025DERORD}\cite{RS2025DERORD} listed 166 solche Produkte.%
\item<3-> Im \citetitle{SE:SO:2024DS}\cite{SE:SO:2024DS} konnten Programmierer Feedback zu 35~Systemen geben.}%
\item<4-> Eine Auswahl der Systeme der jährlichen Developer Surveys von StackOverflow seit 2017.%
\end{itemize}%
%
\locateGraphic[\bracketCite{SE:SO:2024DS}]{4}{width=0.725\paperwidth}{graphics/soDevSurv}{0.1375}{0.22}%
%
\end{frame}%
%
\begin{frame}%
\frametitle{Open Source vs. Proprietäre Software}%
\begin{itemize}
\item Open Source Software~(OSS)\uncover<2->{%
\begin{itemize}%
\item OSS ist Software, deren Quellkode frei verfügbar ist.%
\item<3-> Oftmals wird OSS von Freiwilligen entwickelt, manchmal mit finanzieller Unterstützung von Firmen oder Regierungen.%
\item<4-> OSS kostet kein Geld.%
\item<5-> OSS wird oft auf kollaborativen Plattformen wie \github\ gehosted und ihre Entwicklung wird durch \pglspl{VCS} wie \git\ gemanaged.%
\item<6-> Nach \citeauthor{HNZ2024TVOOSS} war der Wert von OSS im Jahr~\citeyear{HNZ2024TVOOSS} mehr als 8 billionen dollar\cite{HNZ2024TVOOSS}.%
\end{itemize}}%
\item<7-> Proprietäre Software\uncover<8->{%
\begin{itemize}%
\item Proprietäre Software ist kommerzielle Software, die von einem Anbieter entwickelt wird und deren Quellkode normalerweise nicht frei zugänglich ist.%
\item<9-> Solche Software muss gekauft oder gemietet werden.%
\end{itemize}%
}%
\end{itemize}%
\end{frame}%
%
\section{Open Source Relationale Datenbankmanagementsysteme}%
%
\begin{frame}%
\frametitle{Open Source Relationale Datenbankmanagementsysteme}%
\begin{itemize}%
\item Ein großer Teil der relationalen Datenbanken wird von Open Source DBMS gemanaged.%
\item<2-> Diese kosten kein Geld.%
\item<3-> Große Gemeinschaften von Entwicklern existieren um sie herum, entwickeln sie gemeinsam, und können Hilfe leisten.%
\item<4-> Einige dieser Systeme gibt es seit den 1990ern.%
\item<5-> In den frühen 2000ern began ihr großer Aufstieg\cite{P2004OSDMITM}.%
\item<6-> Nun sind sie weitverbreitete und gut getestete Produkte\cite{C20245YOQ}.%
\end{itemize}%
\end{frame}%
%
\begin{frame}%
\frametitle{MySQL}%
\begin{itemize}%
\item \mysql\ ist so ein Open Source DBMS für relationale Datenbanken\cite{WAM2002MRMDFTS,TA2024DDAMWPAM,BT2021HPM,RGS2021BTOTONAMDFPC,D2015LMAM}.%
\item<2-> Es wurde ursprünglich von Michael Widenius und David Axmark in der schwedischen Forma \mysql~AB entwickelt und kam 1996 heraus\cite{C20245YOQ}.%
\item<3-> Es wurd ein wichtiger Teil des weitverbreiteten \lampStack, also einem Systemsetup für Web-Application der auf dem \linux\ Betriebssystem, dem Apache webserver, der \mysql\ Datenbank, und der Server-seitigen Skriptsprache PHP basiert\cite{C2022HAFTLS,H2020ULU2E}.%
\item<4-> Es stand bis ins Jahr~2022 an erster Stelle als beliebtestes DBMS in den Stack Overflow Developer Surveys\cite{SE:SO:2024DS}.
\item<5-> In dem Survey\cite{PMPVEPWGSMB2025ATAODMSTTHOOSP} des Quellkodes von 371~\pgls{Java} Open Source  Projekten auf \github\ wurde \mysql\ auch am häufigsten als DBMS gefunden.%
\item<6-> \mysql~AB wurde im Jahr~2000 von Sun Microsystems gekauft, die wiederum 2010 von Oracle gekauft wurden\cite{C20245YOQ}.%
\item<7-> \mysql\ blieb jedoch OSS.%
\end{itemize}%
\end{frame}%
%
\begin{frame}%
\frametitle{MariaDB}%
\begin{itemize}%
\item Nach dem Aufkauf von Sun durch Oracle gründeten einige der ursprünglichen \mysql-Entwicklet, auch Michael Widenius, den \mysql-Fork \mariadb\cite{R2014MM,B2019LTMEELFFSAA,D2015LMAM,AA2018QAWMV1ITSQ,AA2018QAWMV2IDQ}.%
\item<2-> Sie versprachen, dass \mariadb\ für immer OSS bleiben wird.%
\item<3-> In der Studie\cite{PMPVEPWGSMB2025ATAODMSTTHOOSP} über Datenbanknutzung von \pgls{Java} OSS Programmen war \mariadb\ schon das fünft-populäre relationale DBMS.%
\end{itemize}%
%
\locateGraphic[The \href{https://mariadb.org}{\mariadb} logo is under the copyright of its owners.]{}{width=0.4\paperwidth}{graphics/mariadbLogo}{0.57}{0.137}%
\end{frame}%
%
\begin{frame}[t]%
\frametitle{PostgreSQL}%
\begin{itemize}%
\item \postgresql\cite{TA2024DDAMWPAM,FP2023LP,OH2017PUAR,B2024PELUYDW} ist ein Objekt-relationales DBMS, das auch Konzepte aus der Object-Orientierten Programmierung~(OOP) wie Vererbung unter Tabellen unterstützt.
\item<2-> Es unterstützt auch zusätzliche Datentypen, wie \pgls{JSON}-Objekte und geometrische Daten.
\item<3-> Es stammt aus dem POSTGRES-Projekt, dem Nachfolger des berühmten INGRES-Projekts an der University of Berkeley in Kalifornien\cite{C20245YOQ}.%
\item<4-> Es ist vielleicht das OSS DBMS mit den meisten Features.%
\item<5-> Im \citetitle{SE:SO:2024DS} war es das populärste \pgls{dbms}\cite{SE:SO:2024DS}.
\item<6-> Im Open Source Quellkode-Survey\cite{PMPVEPWGSMB2025ATAODMSTTHOOSP} war es auf Rang~2.
\end{itemize}%
%
\locateGraphic[The \href{https://www.postgresql.org}{\postgresql}~logo is under the copyright of its owners.]{}{width=0.225\paperwidth}{graphics/postgresqlLogo}{0.7}{0.56}%
\end{frame}%
%
\begin{frame}[t]%
\frametitle{PostgreSQL}%
\begin{itemize}%
\item die \pgls{SQL}-Datenbank mit der weitesten Verbreitung ist jedoch \sqlite\cite{WB2019RHSOOS,GPBHKP2022SPPAF,C20245YOQ,HWACIS:HO2023WKUOS}.%
\item<2-> Anders als die vorher genannten System unterstützt es \emph{nicht} die \gls{clientServerArchitecture}.%
\item<3-> Stattdessen wird es als Bibliothek direkt in den Prozess geladen, der es nutzt.%
\item<4-> Heute ist es auf fast jedem Computer, Smartphone, Fernseher, und Automobil installiert und Teil von fast jedem Web Browser.%
\item<5-> Es wurde ursprünglich im Jahr~2000 veröffentlicht und sein Kernentwickler Richard Hipp hat dafür den SIGMOD Systems Award 2017 bekommen\cite{C20245YOQ}.%
\end{itemize}%
%
\locateGraphic[The \href{https://sqlite.org}{\sqlite} logo is under the copyright of its owners.]{}{width=0.4\paperwidth}{graphics/sqliteLogo}{0.57}{0.61}%
\end{frame}%
%
\begin{frame}[t]%
\frametitle{LibreOffice Base}%
\begin{itemize}%
\item Ein DBMS welches mit einzelnen Dateien arbeitet ist auch in der Open Source Bürosoftware \libreoffice\cite{DF2024LTDF,GL2012LTSOOSSCBAFACSOL,S2022L7PFEUU} implementiert: Als \libreofficeBase\cite{FNFHWSKLSSGLFRSRPLJG2022BG7R1BOL7C,S2022L7PFEUU}.%
\item<2-> \libreofficeBase\ ist allerdings mehr als ein DBMS.%
\item<3-> Es bietet eine vielseitige graphische Benutzeroberfläche mit dem man sich auch auf Datenbanken wie \mysql, \mariadb\ und \postgresql\ verbinden kann.%
\item<4-> Mit der Benutzeroberfläche kann man Tabellen, Sichte, Anfragen, Formulare, und Berichte für die Datenbanken erstellen.%
\item<5-> \libreofficeBase\ ist eine freie Alternative zu dem kommerziellen Produkt\microsoftAccess\cite{SSI2023MA2BTA,B2020HOMA2,UC2021AFD}.%
\end{itemize}%
%
\locateGraphic[The \href{https://www.libreoffice.org}{\libreoffice} logo is under the copyright of its owners.]{}{width=0.6\paperwidth}{graphics/libreofficeLogo}{0.2}{0.675}%
\end{frame}%
%
\section{Proprietäre Datenbankmanagementsysteme}%
%
\begin{frame}%
\frametitle{Oracle Database}%
\begin{itemize}%
\item Die erste kommerzielle SQL-Datenbank war die \oracleDB\cite{C20245YOQ,O2007OTHTMIMIOHWCFTPWMIH}.%
\item<2-> Sie ist noch heute eine der erfolgreichsten und weitverbreiteten kommerziellen Datenbanken, mit vielen hochentwickelten Features\cite{BBDDSY2011ADOODM,KK2021EODATASFHPAP}.%
\item<3-> Im Juni~2025 war sie das populärste DMBS überhaupt im Survey\cite{RS2025DERORD}.
\item<4-> In dem \pgls{Java} Quellkodesurvey\cite{PMPVEPWGSMB2025ATAODMSTTHOOSP} war sie auf Rang~4.%
\item<5-> Die Migration von \oracleDB\ nach \postgresql\ ist in\cite{KO2023DMFOTP} diskutiert.%
\end{itemize}%
\end{frame}%
%
\begin{frame}%
\frametitle{Microsoft SQL Server}%
\begin{itemize}%
\item Auf Rang~3 im Survey\cite{RS2025DERORD} stand der \microsoftSqlServer\cite{P2020MSS2ABG,A2024TSAFMSS2,W2018MSSDB}.%
\item<2-> Dieses DBMS hat seine Wurzeln im Jahr~1988, als Microsoft, Ashton-Tate, und Sybase zusammen eine Variante des Sybase SQL-Server für das Betriebssystem IBM~OS/2 entwickelten\cite{W2018MSSDB:TEOMSS}.%
\item<3-> Die erste Version dieses DBMS kam dann 1989 heraus.%
\item<4-> Spätere Versionen unterstützten \microsoftWindows~NT.%
\item<5-> Heute läuft der \microsoftSqlServer\ auch auf \linux~(genau wie die \oracleDB).%
\end{itemize}%
\end{frame}%
%
\begin{frame}%
\frametitle{Microsoft Access}%
\begin{itemize}%
\item Mit \microsoftAccess\cite{LF2022MOSBSO2AM3} bietet Microsoft auch ein DBMS an, das hauptsächlich für einzelne Benutzer auf einem einzelnen Computer gedacht ist.%
\item<2-> Es ist ein unglaublich nützliches und bequemes Werkzeug, das die Features eines relationlen DBMS mit Werkzeugen zum Entwicklen von Formularen und Berichten vereint\cite{SSI2023MA2BTA,B2020HOMA2,UC2021AFD,MM2014RDAMA}.%
\item<3-> Es war sicherlich die Vorlage für \libreofficeBase.%
\item<4-> Die meisten Dinge, die \libreofficeBase\ kann, kann man auch mit \microsoftAccess\ machen, nur oftmals bequemer {\dots} aber \libreofficeBase\ ist kostenfrei\dots%
\end{itemize}%
\end{frame}%
%
\begin{frame}%
\frametitle{IBM DB2}%
\begin{itemize}%
\item IBM's \ibmDB\cite{CWDS2007UDLVWE,BBBCCDMMP2016SPTAFOIDFI} ist ein weiteres frühes relationales DBMS\cite{HS2013THAGOID}.%
\item<2-> Es enstand als Software für mächtige Mainframe-Computer, die IBM selbst hergestellt hat.%
\item<3-> Mainframes sind Computer mit sehr viel Rechenleistung, die oft zentral in großen Organisationen oder Firmen genutzt werden.%
\item<4-> Währen Oracle also Datenbanken für normale Computer hergestellt hat, hat sich IBM auf starke Server konzentriert.%
\item<5-> 1989 hatte die Hälfte aller Mainframe-Kunden \ibmDB\ installiert\cite{HS2013THAGOID}.%
\item<6-> Die \ibmDB\ ist auf Rang~6 im \citetitle{RS2025DERORD}\cite{RS2025DERORD} und auf Rang~23 im \citetitle{SE:SO:2024DS}\cite{SE:SO:2024DS}.%
\end{itemize}%
\end{frame}%
%
\section{Software in diesem Kurs}%
%
\begin{frame}%
\frametitle{Dieser Kurs}%
\begin{itemize}%
\item Das Ziel dieses Kurses ist eine \emph{praktische} Einführung in Datenbanken.%
\item<2-> Daher werden wir mit praktischen Beispielen arbeiten und (mehr oder weniger) realistische Datenbanken auf echten Datenbankmanagementsystemen erstellen.%
\item<3-> Das bedeuted, dass wir verschiedene interessante Themen behandeln werden.%
\item<4-> Und in allen Themen werden wir echte praktische Erfahrung sammeln.%
\item<5-> Und für die meisten Themen werden wir ein passendes Softwarewerkzeug verwenden.%
\end{itemize}%
\end{frame}%
%
\begin{frame}[t]%
\frametitle{Datenbankmanagementsystem:~PostgreSQL}%
\begin{itemize}%
\item Wir werden mit einem echten DBMS arbeiten.%
\item<2-> Wir haben uns für \postgresql\cite{TA2024DDAMWPAM,FP2023LP,OH2017PUAR,B2024PELUYDW} entschieden.%
\item<3-> DBMSe sind normalerweise Server an die man SQL-Befehle mit Hilfe eines Clientprograms vom Terminal aus schicken kann.%
\item<4-> Dafür nehmen wir den \psql-Klient, der zu \postgresql\ dazu gehört.%
\end{itemize}%
\locateGraphic[The \href{https://www.postgresql.org}{\postgresql}~logo is under the copyright of its owners.]{}{width=0.225\paperwidth}{graphics/postgresqlLogo}{0.7}{0.54}%
\locateGraphic{4}{width=0.5\paperwidth}{graphics/psqlExample}{0.08}{0.49}%
\end{frame}%
%
\begin{frame}[t]%
\frametitle{GUI:~LibreOffice Base}%
\begin{itemize}%
\item DBMS werden oft als Server installiert, dem man mit einem Klienten über die Kommandozeile SQL-Kommandos schicken kann.%
\item<2-> Es gibt aber auch Werkzeuge, die eine \pgls{GUI} anbieten, mit denen man Formulare und Berichte entwerfen kann.
\item<3-> Formulare sind strukturierte Fenster / Dialoge / Masken zur Dateneingabe.%
\item<4-> Berichte sind Dokumente, die automatisch mit den Daten aus einer Datenbank gefüllt werden.%
\item<5-> Wir werden auch so ein Werkzeug ausprobieren: \libreofficeBase\cite{FNFHWSKLSSGLFRSRPLJG2022BG7R1BOL7C,S2022L7PFEUU}.%
\end{itemize}%
%
\locateGraphic[The \href{https://www.libreoffice.org}{\libreoffice} logo is under the copyright of its owners.]{}{width=0.6\paperwidth}{graphics/libreofficeLogo}{0.2}{0.675}%
\end{frame}%
%
\begin{frame}[b]%
\frametitle{Zugriff über Programmiersprache}%
\begin{itemize}%
\item Ein DBMS ist normalerweise das Backend der Infrastruktur einer Organisation.%
\item<2-> Nutzer arbeiten nur sehr selten direkt mit einem DBMS.%
\item<3-> Stattdessen werden oft mehrere Applikationen entwickelt, die auf die Datenbank mit Hilfe einer \pgls{API} zugreifen.%
\item<4-> Wir wollen auch das ausprobieren.%
\item<5-> Dafür werden wir die Programmiersprache \python\cite{K2018EIPFEUU,A2002PC,H2023ABGTP3P,LH2015DSAAWP,programmingWithPython} verwenden.%
\item<6-> Dazu gibt es eine Bibliothek für den Zugriff auf \postgresql, nämlich \psycopg\cite{VDGE2010P}.%
\end{itemize}%
\locateGraphic[The \href{https://www.python.org}{\python\ programming language} logo is under the copyright of its owners.]{}{width=0.5\paperwidth}{graphics/pythonLogo}{0.1}{0.15}%
\locateGraphic[Copyright \textcopyright~Gabriella Albano and the Psycopg team]{}{width=0.15\paperwidth}{graphics/psycopgLogo}{0.675}{0.1}%
\end{frame}%
%
\begin{frame}[b]%
\frametitle{Konzeptuelle Modelle}%
\begin{itemize}%
\item Ein wichtiger Schritt im Datenbankentwicklungsprozess ist das Erstellen von abstrakten konzeptuellen Modelle der Problemdomäne.%
\item<2-> Dazu werden graphische Diagramme, sogenannte \glsreset{ERD}\pglspl{ERD} gezeichnet, die die Objekte und deren Beziehungen aus der realen Welt darstellen, die wir in der Datenbank modellieren wollen.%
\item<3-> Diese konzeptuellen Modelle sind unabhängig von der letztendlich verwendeten DBMS-Technologie.%
\item<4-> Als Werkzeug um solche Diagramme zu zeichnen verwenden wir \yEd\cite{SG2015MDAWY,Y2011YGEM}.%
\end{itemize}%
\locateGraphic[%
\parbox{0.6\paperwidth}{\strut\\[10pt]The logo of the \href{https://www.yworks.com/products/yed}{\yEd\ graph editor}. %
The \yEd~logo is protected by copyright. %
\yEd~is a registered trademark of \href{https://www.yworks.com}{yWorks~GmbH}. %
Unauthorized use, reproduction, or distribution is strictly prohibited.}%
]{}{width=0.2\paperwidth}{graphics/yEdLogo}{0.6}{0.06}%
\end{frame}%
%
\begin{frame}[b]%
\frametitle{Logische Modelle}%
\begin{itemize}%
\item Die konzeptuellen Modelle müssen letztendlich auch logische Schemas abgebildet werden, die auf einem spezifischen Datenmodell und einer spezifischen Datenbanktechnologie beruhen.%
\item<2-> Die logischen Modelle beschreiben, wie Anwendungen und Nutzer die Daten sehen.%
\item<3-> Sie können z.B. in SQL implementiert werden.%
\item<4-> Sie können auch visuell mit Werkzeugen modelliert werden, als Diagramme die so ähnlich wie ERDs aussehen, die aber an eine bestimmte Datenbanktechnologie gebunden sind.%
\item<5-> Wir verwenden den \pgmodeler\cite{AES2006PPDM}, mit dem solche logischen Modelle für \postgresql\ erstellt werden können.%
\end{itemize}%
\locateGraphic[%
The \href{https://pgmodeler.io}{\pgmodeler\ logo} is under the copyright of Raphael Araújo~e~Silva.%
]{}{width=0.2\paperwidth}{graphics/pgmodelerLogo}{0.6}{0.06}%
\end{frame}%
%
\begin{frame}%
\frametitle{Viele Werkzeuge}%
\begin{itemize}%%
\item Das sind viele Werkzeuge.%
\end{itemize}%
\uncover<2->{%
\bestPractice{manyTools}{%
Ein(e) Informatiker(in) ist immer in der Lage neue Werkzeuge zu erlernen und tut dies auch gerne.%
\uncover<3->{ Ein(e) Informatiker(in) muss in der Lage sein, duzende oder hunderte verschiedene Softwarewerkzeuge zu nutzen, jeweils für ihre spezifischen Aufgaben.%
\uncover<4->{ Ein(e) Softwareingenieur(in) ist ein(e) Handwerker(in) und seine/ihre Kenntnis über verschiedene Software ist sein/ihr Werkzeuggürtel.}}%
}}%
\end{frame}%
%
\begin{frame}%
\frametitle{Spezifische Werkzeuge}%
\begin{itemize}%
\item Wir nutzen ein spezifisches DBMS, spezifische Visualisierungstools, eine spezifische Programmiersprache und eine spezifische Bibliothek, um auf das DBMS zuzugreifen.%
\item<2-> Ist das eine gute Idee?%
\item<3-> Wenn Sie jemals mit Datenbanken in Ihrem professionellen Leben arbeiten werden, dann wahrscheinlich mit einem anderen DBMS und anderen Werkzeugen.
\item<4-> Wird die Kenntnis über die spezifischen Werkzeuge, die wir hier verwenden, Ihnen dann irgendetwas Nutzen?%
\end{itemize}%
\end{frame}%
%
\begin{frame}%
\frametitle{Spezifische Werkzeuge}%
\begin{itemize}%
\item Wird Ihnen das jemals etwas nutzen?\uncover<2->{~Ja.}%
\item<3-> Erstens diskutieren wir die Grundlagen relationaler Datenbanken und die Werkzeuge sind nur die Beispiele.%
\item<4-> Theoretisches Wissen ist nur begrenzt nützlich in einem praktischen Szenario, wenn man es nicht praktisch umsetzen kann.%
\item<5-> Es hilft gar nichts, die Grundlagen von Datenbanken zu kennen, wenn Sie sich nicht mit einem Terminal zum DBMS verbinden und SQL-Kommandos dorthin senden können, wenn Sie nicht wissen, wie man Fehlermeldungen liest, wenn Sie niemals mit einem echten DBMS gearbeitet haben.%
\item<6-> Sie müssten all das lernen {\dots} \emph{unter Zeitdruck.}%
\item<7-> Wenn Sie aber z.B.\ \postgresql\ schonmal installiert haben, auf Ihrem Computer, schonmal SQL-Befehle dorthin gesendet haben, schonmal mit \postgresql-Fehlermeldungen arbeiten mussten\dots%
\item<8-> {\dots}dann können Sie sicherlich auch sehr schnell lernen, wie man das mit \mysql, \mariadb, oder anderen DBMSen macht.%
\end{itemize}%
\end{frame}%
%
\begin{frame}%
\frametitle{Linux}%
%
\bestPractice{knowLinux}{%
Jede(r) professionelle Informatiker(in), Softwareentickler(in), Softwareingenieur(in), Datenbankadministrator(in), oder Systemadministrator(in) sollte mit dem \linux\ Betriebssystem umgehen können.%
}%
\begin{itemize}%
\item Vielleicht können Sie ja mit dem einfach zu nutzenden \ubuntu\ \linux\cite{CN2020ULB,H2020ULU2E} anfangen.%
\item<2-> Wenn Sie \microsoftWindows\cite{B2023W1IO} nutzen, vielleicht können Sie \ubuntu\ ja in einer virtuellen Maschine installieren.%
\item<3-> Ich empfehle jedoch ausdrücklich \linux\ zu lernen und zu nutzen.%
\end{itemize}%
\end{frame}%
%
\section{Literatur}%
%
\begin{frame}%
\frametitle{Literatur}%
\begin{itemize}%
\item Für diesen Kurs sollten Sie unser Kursbuch lesen:~\furtherReading{databases}.%
\item<2-> Um weitere Informationen zu erhalten, empfehle ich das Anschauen der Slides von anderen Vorlesungen über Datenbanken, die online verfügbar sind.%
\item<3-> Weiterhin gibt es viele gute Bücher über Datenbanken, von denen Sie sicherlich einige in der Bibliothek finden können, vielleicht sogar in chinesischer Übersetzung.%
\item<4-> Es gibt auch Webseiten, auf denen man nützliche Informationen finden kann.%
\end{itemize}%
\end{frame}%
%
\begin{frame}%
\frametitle{Andere Vorlesungen}%
\begin{itemize}%
\item \furtherReading{S2024D},%
\item \furtherReading{KC2024DS},%
\item \furtherReading{P2024C6DS},%
\item \furtherReading{T2025CDBMS},%
\item \furtherReading{C2016CDMS},%
\item \furtherReading{R2024CDS}.%
\end{itemize}%
\end{frame}%
%
\begin{frame}%
\frametitle{Andere Vorlesungen}%
\begin{itemize}%
\item \furtherReading{G2011EW2ITDS},%
\item \furtherReading{P2006CITRD},%
\item \furtherReading{SS2005EIDDDFDB},%
\item \furtherReading{V1999C5DMS}.%
\end{itemize}%
\end{frame}%
%
\begin{frame}%
\frametitle{Generelle Bücher~(1)}%
\begin{itemize}%
\item \furtherReading{PW2022APITD},%
\item \furtherReading{SKS2019DSC},%
\item \furtherReading{HRT2021MDM},%
\item \furtherReading{EN2015FODS},%
\item \furtherReading{KE2015DEE},%
\item \furtherReading{TATPC2009TMGTD},%
\item \furtherReading{H2020DDFMM2AE},%
\end{itemize}%
\end{frame}%
%
\begin{frame}%
\frametitle{Generelle Bücher~(2)}%
\begin{itemize}%
\item \furtherReading{K2017SAATPODP},%
\item \furtherReading{CM2018DSDIM},%
\item \furtherReading{D2003AITDS},%
\item \furtherReading{RG2002DMS},%
\item \furtherReading{H2016RDDAI},%
\item \furtherReading{GMUW2008DSTCB},%
\end{itemize}%
\end{frame}%
%
\begin{frame}%
\frametitle{Generelle Bücher~(3)}%
\begin{itemize}%
\item \furtherReading{GMUW2008DSTCB},%
\item \furtherReading{HM2024IMARD},%
\item \furtherReading{BHS2015RIDS},%
\item \furtherReading{MS2001S1URLC},%
\item \furtherReading{CM2000MDMAUDA},%
\item \furtherReading{T2018ISARD},%
\item \furtherReading{SPMP1998SI2TDDASVEI12T}.%
\end{itemize}%
\end{frame}%
%
\begin{frame}%
\frametitle{Bücher zu Spezifischen Werkzeugen~(1)}%
\begin{itemize}%
\item \furtherReading{FP2023LP},%
\item \furtherReading{TA2024DDAMWPAM},%
\item \furtherReading{P2020MSS2ABG},%
\item \furtherReading{A2024TSAFMSS2},%
\item \furtherReading{KK2021EODATASFHPAP},%
\item \furtherReading{MM2014RDAMA},%
\end{itemize}%
\end{frame}%
%
\begin{frame}%
\frametitle{Bücher zu Spezifischen Werkzeugen~(2)}%
\begin{itemize}%
\item \furtherReading{UC2021AFD},%
\item \furtherReading{SSI2023MA2BTA},%
\item \furtherReading{B2020HOMA2},%
\item \furtherReading{CWDS2007UDLVWE},%
\item \furtherReading{BBBCCDMMP2016SPTAFOIDFI}.%
\end{itemize}%
\end{frame}%
%
\begin{frame}%
\frametitle{Webseiten}%
\begin{itemize}%
\item \furtherReading{PGDG:PD},%
\item \furtherReading{SE:DA},%
\item \furtherReading{B2025DS},%
\item \furtherReading{D2022DN},%
\item \furtherReading{W2018MSSDB}.%
\end{itemize}%
\end{frame}%
%
\section{Zusammenfassung}%
%
\begin{frame}%
\frametitle{Zusammenfassung}%
\begin{itemize}%
\item Es gibt eine Vielzahl an relationalen Datenbankmanagementsystemen.%
\item<2-> Manche sind Open Source, was gut ist.%
\item<3-> Andere sind proprietär und kosten Geld.%
\item<4-> In diesem Kurs nutzen wir nur kostenfreie Produkte.%
\item<5-> Und zwar ziemlich viele.%
\item<6-> Es gibt viele Quellen, von denen Sie zusätzliche Informationen erhalten können.%
\end{itemize}%
\end{frame}%
%
\endPresentation%
\end{document}%%
\endinput%
%
